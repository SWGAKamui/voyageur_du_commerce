\chapter*{Introduction}

\begin{quotation}
\textit{Le problème du voyageur de commerce est un problème mathématique qui consiste, étant donné un ensemble de villes séparées par des distances données, à trouver le plus court chemin qui relie toutes les villes. Il s'agit d'un problème d'optimisation pour lequel on ne connait pas d'algorithme permettant de trouver une solution exacte en un temps polynomial. De plus, la version décisionnelle de l'énoncé (pour une distance D, existe-t-il un chemin plus court que D passant par toutes les villes ?) est connue comme étant un problème NP-complet.}
\end{quotation}
\begin{flushright}
Problème du voyageur de commerce - Wikipédia
\end{flushright}

\par
Nos travaux, présentés dans ce rapport, concerne les méthodes de résolutions de ce problème. On parle des algorithmes et de leurs implémentations, mais aussi des outils de travail en équipe et autres méthodes de résolution des problèmes.