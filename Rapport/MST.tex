\chapter{Algorithme d'approximation - \textit{Minimum Spanning Tree}}


Dans le cas du \textit{Minimum Spanning Tree}, plusieurs algorithmes existent déjà. Il fallait dans un premier temps définir quel algorithme utiliser.

\section{Algorithme}

Le choix s'est fait entre les algorithmes suivants :
\begin{itemize}
	\item L'algorithme de Prim
	\item L'algorithme de Kruskal
	\item L'algorithme de Borůvka
\end{itemize}

Parmi ces trois algorithmes, celui retenu est celui de Kruskal. Beaucoup plus simple à coder et tout aussi efficace.
L'algorithme de Kruskal consiste à choisir les arêtes de poids le plus faible sans qu'aucun cycle ne soit créé. Le résultat est un arbre, comprenant tous les sommets, de poids minimal.

\begin{algorithm}
\caption{Algorithme de Kruskal}
\begin{algorithmic}


\Function{trouverRouteOptimale}{ref G : graphe, ref poidsTab : matrice de décimaux}

\State $E : ensemble vide$
\State $a : arete$

\State \Call{trierAreteOrdreCroissant}{G, poidsTab}
\State $a \gets \Call{prendreProchaineArete}{G}$
\While {$a != NULL$}
	\If {$!(\Call{creeUnCycle}{E, a})$}
		\State \Call{ajouterArete}{E, a}
	\EndIf
	\State $ a \gets \Call{prendreProchaineArete}{G}$
\EndWhile
\State \Return $E$
\EndFunction
\end{algorithmic}
\end{algorithm}

\section{Implémentation}

Pour pouvoir utiliser les "arêtes", il a fallu créer une structure \textit{Edge} qui prend trois int : deux pour les sommets et un autre pour contenir le poids de l'arête.

\subsection{Implémentation simple de l'algorithme}

Ensuite, il a fallut implémenter l'algorithme de Kruskal. Pour celà, il faut d'abord commencer par appeler la fonction \textit{parcoursMST()}. Celle-ci va, à son tour, appeler la fonction \textit{readMatrixAndCreateEdges()} qui permet de créer un tableau contenant toutes les arêtes possibles. Le tableau, pendant son remplissage, est automatiquement trié.

La fonction va alors récupérer les arêtes via \textit{getNextEdge()}. Celle-ci récupère les arêtes du tableau triées par ordre de poids croissant, et vérifies à l'aide de \textit{isCyclic()} si, en ajoutant la nouvelle arête à la route, un cycle se crée.

\subsection{Implémentation de la recherche de cycle}

L'algorithme utilise la recherche de cycle dans un graphe. Pour effectuer cette recherche, deux fonctions sont utilisées :
\begin{itemize}
	\item \textit{isCyclic()}
	\item \textit{searchRecur()}
\end{itemize}

La première fonction est la fonction de base. Elle commence par ajouter l'arête à tester dans la route, puis elle va tester tous les chemins possibles à partir des arêtes contenus dans la route. 

La recherche est effectuée comme si le graphe était non orienté. Dans le cas d'une arête \textit{(u,v)}, la fonction va vérifier une à une toutes les villes possibles si il existe une arête avec \textit{u} dans la route étant la ville, puis de même avec \textit{v}.
Si on trouve une arête répondant à ce critère, on considère que la ville est traversée, puis on fait appel à \textit{searchRecur()} pour tester toutes les possibilités de chemin commençant par la précédente arête.

La deuxième fonction suit le même raisonnement que la première, excepté qu'elle est récursive et qu'elle vérifie que la prochaine ville qui va être visité ne l'a pas déjà été. 

\section{Complexité}

La première fonction va être effectuée, dans le pire des cas, le nombre d'arêtes de la matrice. Une optimisation a été effectuée, qui a été de prendre uniquement les arêtes au-dessus de la diagonale de 0 dans la matrice. Soit \textit{t} la taille de la matrice, on arrive donc à :
\begin{equation}
nbAretesPossibles = t^2 \rightarrow \overset{t}{\underset{i=0}{\sum}}i \rightarrow \frac{n(n-1)}{2}
\end{equation}
De plus, on va vérifier à chaque fois si un cycle existe en ajoutant l'arête. Ce qui signifie que nous obtenons :
\begin{equation}
Complexite = nbAretesPossibles = nbVerificationCycles
\end{equation}
Pour chaque vérification de cycle, on effectue au maximum \textit{i} fois le nombre de valeurs de villes possibles, et \textit{i}*\textit{j} fois le nombre d'arête dans la route actuelle. On trouve donc :
\begin{equation}
Complexite = nbVerificationCycles * nbVillesPossibles * nbAreteActuelles
\end{equation}
De la même façon, \textit{searchRecur()} peut être appelée au maximum \textit{i} fois le nombre d'arêtes dans la route, et à chaque appel, elle appelle \textit{i} fois via la boucle \textit{for}. Ce qui donne :
\begin{equation}
Complexite = nbVerificationCycles * nbVillesPossibles * nbAreteActuelles^3 
\end{equation}
En considérant que le nombre d'arêtes actuellement dans la route est quasiment à son maximum, soit \textit{n} le nombre d'arête de la route, et \textit{v} le nombre de villes possibles, on trouve finalement que, au maximum, la complexité est de :
\begin{equation}
Complexite = v * n^3 * \frac{n(n-1)}{2}
\end{equation}
On peut noter que le nombre de villes possible est équivalent au nombre d'arêtes de la route, d'où, dans le \underline{pire} des cas :
\begin{equation}
Complexite = n * n^3 * \frac{n(n-1)}{2} = O(n^5)
\end{equation}
Cette complexité est néanmoins améliorable, en supprimant notamment l'utilisation de la variable \textit{nbVillesPossibles} dans la fonction \textit{isCyclic()}.