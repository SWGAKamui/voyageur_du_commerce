\chapter{Gestion du projet avec Git}

\par\bigskip
Git est un logiciel de versions décentralisé. Il permet de synchroniser les différentes versions d'un projet, et de travailler en groupe sur un même projet.
\par\bigskip
L'hébergement du projet est sur \href{https://github.com}{GitHub}, qui propose des hébergements gratuits de répertoires Git. Notre répertoire de projet est disponible \href{https://github.com/Lumi-Bjorn/Edd_project}{à cet endroit} : \url{https://github.com/Lumi-Bjorn/Edd_project}.
\par\bigskip

\section{Avantages}
\begin{description}
	\item[Gestion des branches]Lors de la création d'une branche, \textit{Git} ne recrée pas intégralement l'arborescence du projet, permettant d'avoir une économie de temps et d'espace pour des projets denses.
	\item[Fusion efficace]\textit{Git} sait s'adapter à des modifications d'un même fichier par plusieurs utilisateurs, et ainsi de faciliter les démarches de fusion de branches.
	\item[Répertoire local]\textit{Git} copie le répertoire distant, ce qui permet de pouvoir travailler sur son projet, même sans accès à Internet.
	\item[Rapidité]\textit{Git} travaille sur le répertoire local. De plus, les mises à jour du répertoire distant sont extrêmement rapides.
\end{description}
\section{Défauts}
\begin{description}
	\item[Complexité]Une même commande \textit{Git} peut être utilisée pour effectuer 2 tâches différentes. Aussi, les commandes ne sont pas très intuitives comparées à \textit{SVN}.
	\item[Beaucoup de commandes] \textit{Git} possède beaucoup de commandes. Cependant, pour un usage quotidien "classique", ce nombre reste assez faible.
\end{description}

\par\bigskip
Comparé à \textit{SVN}, \textit{git} jouit d'une simplicité pour développer plusieurs fonctions instantanément, ainsi que de la possibilité de travailler sur son répertoire local et d'effectuer une simple mise à jour lors de la récupération d'Internet. C'est pour ces différentes raisons que \textit{Git} a été choisi.

