\chapter{Module de lecture des fichiers TSP}

Afin de récupérer des matrices différentes, il est nécessaire d'implémenter un module pour lire les fichiers TSP. 

\section{Algorithme}

\begin{algorithm}
\caption{Algorithme pour lire un fichier}
\begin{algorithmic}

\Function{recupererMatrice}{ref nomFichier : chaîne de caractères}

\State $file \gets \Call{ouvrirFichier}{nomFichier}$
\State $buffer \gets \Call{lireFichier}{}$
\While {$buffer != "DIMENSION"$}
	\State $buffer \gets \Call{lireFichier}{}$
\EndWhile

\State $dimension \gets \Call{lireProchaineValeur}{}$

\While{$buffer != "EDGE\_WEIGHT\_SECTION"$} 
	\State $buffer \gets \Call{lireFichier}{}$
\EndWhile

\State $matrice \gets \Call{creerMatrice}{dimension}$

\While{$buffer != "DISPLAY\_DATA\_SECTION"$}

	\Call{ajouterDansMatrice}{buffer}
\EndWhile

\Return $matrice$
\EndFunction
\end{algorithmic}
\end{algorithm}

\section{Utilisation}

Pour utiliser le module, il suffit donc de faire passer le chemin du fichier à ouvrir à la fonction \textit{fopen()}. 
De plus, une autre fonction est disponible dans le module, et qui permet de lire les coordonnées des villes placées à la fin du fichier. Ceci est notamment utile pour l'interface graphique qui a besoin des coordonnées des villes pour fonctionner.

