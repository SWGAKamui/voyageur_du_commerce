\chapter{Heuristique simple - \textit{Nearest Neighbour}}

\section{Algorithme}

\begin{algorithm}
\caption{Algorithme par \textit{Nearest Neighbour}}
\begin{algorithmic}

\Function{trouverRoute}{ref distanceVilles : matrice de décimaux}

\State $i \gets 0$
\State $tabChemin [0...nbVilles]$ 
\State $indiceTabChemin \gets 0$

\While{$!\Call{isEndOf}{tabChemin}$}
	\State $distanceMin \gets 2*nbVilles$
	\State $indiceMin \gets 0$
	\State $colonne \gets 0$
	\While{$colonne < nbVilles$}
		\If{$(distanceVille[i][colonne] < distanceMin) \wedge (i!=colonne) \wedge (colonne+1\not\in tabChemin[0...indiceTabChemin][0...indiceTabChemin])$}
			\State $distanceMin \gets distanceVilles[i][colonne]$
			\State $incideMin \gets colonne$
		\EndIf		
		\State $colonne++$
	\EndWhile
	\State $tabChemin[indiceTabChemin]\gets indiceMin +1$
	\State $indiceTabChemin++$
	\State $i\gets indiceMin$
\EndWhile

\Return $tabChemin$
\EndFunction
 \end{algorithmic}
\end{algorithm}

L'algorithme se base principalement sur la recherche du point le plus proche.
On part du premier point, on récupère le plus proche, puis on récupère le plus proche de ce dernier, et ainsi de suite ...

\section{Implémentation et problèmes}

L'implémentation du précédent algorithme est présent à l'annexe C.
\par\bigskip
Nous devions implémenter, lors du premier TP, la fonction \textit{NearestNeighbour}. Mais la fonction ne fonctionnait pas car, avant même de concevoir la partie algorithmique simple de \textit{NearestNeighbour}, nous cherchions déjà à essayer d'utiliser des fonctions de gestion de matrices.

C'est pourquoi, \textit{NearestNeighbour} a été développé en utilisant une matrice rentrée en brut, ainsi que le test renvoyant le chemin (chemin exact car vérifié « manuellement »).
Mais, à ce moment là, tout était encore codé en un unique fichier(header, client, fournisseur), et il a été difficile séparer tous les fichiers, car un soucis de déclaration de variables persistait (impossible de savoir s'il y avait une histoire de variable globale ou non). Cependant, après une intense phase de débogage, le problème a fini par être résolu.

L'implémentation de \textit{NearestNeighbour} a ensuite été changée, de manière à utiliser le module \textit{Matrix} dans le code. La gestion de la mémoire est bonne sur le module \textit{NearestNeighbour}, toute mémoire allouée est libérée (\textit{vérifié par valgrid ./TestNearestNeighbour}).

\section{Test}

Le test de l'algorithme s'effectue via la comparaison entre le tableau de résultats qui devrait être trouvé et le tableau retourné par l'algorithme.

\lstset{style=customc}
\lstinputlisting[caption=TestNearestNeighbour.c - Lignes 30 à 33, firstline=30, lastline= 33]{/home/amolith/Edd/Projet/Git/Edd_project/NearestNeighbour/TestNearestNeighbour.c}
